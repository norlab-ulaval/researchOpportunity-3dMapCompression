\documentclass[10pt,letterpaper,oneside]{article}

\input{./latexGoodPractices/preamble}

%----------------------------------------
% FILL THIS SECTION

\newcommand{\projectTitle}{3D Map Compression based on Differential Geometry}

% Internship Project, Master's Project, Doctoral Project
\newcommand{\projectLevel}{Master's Project}

\newcommand{\projectSupervisor}{Prof. F. Pomerleau}
\newcommand{\projectSupervisorEmail}{francois.pomerleau@ift.ulaval.ca}

\author{Fran\c{c}ois Pomerleau \\
       Laval University\\
       1065, av. de la Médecine \\
       Quebec, Qc \\
       Canada G1V 0A6 \\
       \texttt{<francois.pomerleau@ift.ulaval.ca>}
%       \and
%       Somebody Else \\
%       Laval University\\
%       1065, av. de la Médecine \\
%       Quebec, Qc \\
%       Canada G1V 0A6 \\
%       \texttt{<Somebody.else@ulaval.ca>}
}

% Change to your specific file
\addbibresource{./references.bib}

% ---------------------------------------------------------------
% Load style
\input{researchOpportunityStyle.tex}
% ---------------------------------------------------------------

% Wrap text around a figure
\usepackage{wrapfig}
\setlength\intextsep{0pt}

% Fill the template with text
\usepackage{lipsum}

\acrodef{ICP}{iterative closest point}

%================================================================
\begin{document}
\makeCustomTitle

% ---------------------------------------------------------------
\section*{Project Proposal}

\begin{wrapfigure}{R}{0.35\textwidth}
\centering
\includegraphics[width=0.35\textwidth]{./figs/overview.pdf}
\caption{
Dense 3D map generated from a Velodyne HDL-64E mounted on a mobile robot and containing more than a million points.
}
\label{fig:overview}
%\vspace{-20pt}
\end{wrapfigure}

Prior work on lidar-based registration algorithm \cite{Pomerleau2014} have been recently used to create larger and larger 3D maps.
The map of one of the satellite campus of the University of Toronto (see \autoref{fig:overview}) is one example where only a few hours of data collection lead to a number of points at the limit of real-time computation capability.
Although nice looking, dense maps are not necessarily tailored for accurate registration.
Moreover, filters reducing the number of points and aiming at a uniform density help be mitigate the radial distribution generated by the sensor, but cause problem when the size of the environment change drastically (i.e., when a robot move from indoor to outdoor environments).

The goal of this project is to investigate geometric structures (e.g., line, plane, curvature) as a way to reduce the number of points necessary for an accurate localization.
Those structures should be incorporated in the \ac{ICP} algorithm, where the objective function can adapt to those informations.
As a starting point, more information on differential geometry and \ac{ICP} can be found in \cite{Pomerleau2015}.
The final implementation is expected to be implemented in C++ and integrated within \texttt{libpointmatcher} (\url{https://github.com/ethz-asl/libpointmatcher}), a modular library containing multiple algorithms used inside \ac{ICP} and currently used in multiple research projects around the world.
Proof of concept will be demonstrated on existing data sets.
Final experiments will be conducted on a mobile robot equipped with a Velodyne HDL-32e in multiple environments ranging from underground tunnels to forests.

% ---------------------------------------------------------------
\section*{Research Environment}

The project will be hosted by the Northern Robotics Laboratory (norlab) located on the main campus of Laval University.
The university was established in \num{1663}, making it the oldest academic institution in Canada and the first school in North America to offer higher education in French.
It currently enrolls \num{50000} students, from which around \num{9000} are at the postgraduate level.
Norlab is specialized in mobile and autonomous systems working in winter or difficult conditions. 
We aim at investigating new challenges related to navigation algorithms to push the boundary of what is currently possible to achieve with a mobile robot in real-life conditions. 
The current focus of the laboratory is on localization algorithms designed for laser sensors (lidar) and 3D reconstruction of the environment.

% ---------------------------------------------------------------
\printbibliography


\end{document}
